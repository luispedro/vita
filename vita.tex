\documentclass{article}
\usepackage{textcomp}
\usepackage{graphicx}
\usepackage{xcolor}

\definecolor{lightg}{HTML}{999999}
\definecolor{medg}{HTML}{666666}
\definecolor{darkg}{HTML}{333333}

\usepackage{wrapfig}
\usepackage{cmbright}
%FONTS
\usepackage{fontspec}
\font\head="Qlassik Medium:letterspace=4" at 34pt % http://www.dafont.com/qlassik.font
\font\headtwo="Qlassik Medium:letterspace=4" at 24pt % http://www.dafont.com/qlassik.font
\font\subhead="Avdira Italic" at 13pt
\font\subsubhead="Avdira Italic" at 11pt
\font\Quote="Linux Libertine O Bold Italic:mapping=tex-text" at 16pt
\font\Contact="Linux Libertine O" at 11pt
\font\Text="Linux Libertine O:mapping=tex-text" at 11pt
\font\Textit="Linux Libertine O/I:+onum" at 11pt
\font\CvItem="Lucida Sans Regular:mapping=tex-text" at 10pt
\font\Date="Lucida Sans Demibold Roman:mapping=tex-text" at 9pt

\renewcommand\section[1]{%
    \par\vspace{1.4em}\penalty-100%
    {\subhead #1}%
    \par\penalty100\vspace{0.1em}\penalty100%
}
\renewcommand\subsection[1]{%
    \par\vspace{.1em}%
    {\hspace{1em}\subsubhead #1}%
    \par\vspace{.2em}%
}
\newcommand\cvitem[2][\relax]{%
    \par\vspace{.4em}
    \if\relax#1\else{\Date \textcolor{medg}{#1}}\hspace{1em}\fi%
    {\CvItem #2}%
    \par%
}
\newcommand\showdoi[1]{%
    \href{http://dx.doi.org/#1}{[DOI]}%
}
\newcommand\preprint[1]{%
    \href{#1}{[Preprint]}%
}
\newcommand\ignore[1]{\relax}

\usepackage[margin=1in]{geometry}
\usepackage[xetex,bookmarks,colorlinks,breaklinks]{hyperref}

\usepackage{enumitem}
\renewcommand{\labelitemi}{\textcolor{lightg}{\symbol{"F6B9}}}
\pagestyle{empty}
\begin{document}

\parindent=0cm

\begin{minipage}[t]{0.60\linewidth}%
\head LUIS PEDRO COELHO\\
\headtwo \textcolor{darkg}{Curriculum Vitæ}

\end{minipage}
\hfill
\begin{minipage}[t]{0.34\textwidth}%
\vspace{-2.2em}
{\Contact%
\textcolor{medg}{%
European Molecular Biology\\
Laboratory (EMBL)\\
Meyerhofstraße, Heidelberg\\
Germany}\\
\textcolor{darkg}{phone} :  +49 (0) 6221 387 (8296)\\
\textcolor{darkg}{email} : luis@luispedro.org\\
\textcolor{darkg}{Erdös-Bacon Nr.} : 7

\textcolor{darkg}{Citizenship} : U.K.\ and Portugal
}
\end{minipage}

\vspace{2.3em}

\section{Education}
\Text

\cvitem[2011]{PhD in computational biology, Carnegie Mellon University}
Dissertation topic: \emph{Modeling the Space of Subcellular Location Patterns
Using Images and Other Sources of Information}, advised by Prof.\ Robert F.
Murphy.

\cvitem[2006]{MS in computer science, Instituto Superior Técnico (Technical University Lisbon)}
Dissertation topic: \emph{Bayesian Network Parameter Estimation Using Noisy
Observations or Soft Evidence}, advised by Prof.\ Arlindo Oliveira.

\cvitem[2004]{BS in computer science, Instituto Superior Técnico (Technical University Lisbon)}
Finished top of my class.

\section{Professional Experience}

\cvitem[2013--Present]{Postdoctoral researcher at European Molecular Biology Laboratory (EMBL)}
Supervisor: Dr.\ Peer Bork

\cvitem[2012]{Postdoctoral researcher at Instituto de Medicina Molecular (Lisbon)}
Supervisor: Dr.\ Musa Mhlanga

\section{Scholarships \&\ Awards}

\cvitem[2013]{JBS Authors' Choice Award}
For most popular article published in The Journal of Biomolecular Screening in 2012.

\cvitem[2012]{Siebel Scholar}
Awarded annually for academic excellence and demonstrated leadership to 85 top
students from the world's leading graduate schools

\cvitem[2007--2011]{PhD. Scholarship from Portuguese Science Foundation}
\cvitem[2009]{Joint CMU-U. of Pittsburgh PhD.\ in Computational Biology Research Excellence Award}
\cvitem[2008]{Joint CMU-U. of Pittsburgh PhD.\ in Computational Biology Academic Excellence Award}
\cvitem[2006]{Fulbright Fellow}
Awarded for Portuguese citizens to study in the United States. Circa~12 are
awarded annually from hundreds of applicants.

\cvitem[2005]{Scholarship from Portuguese Science Foundation}
\cvitem[2004]{Second Prize in \emph{Lisboa à Letra} short story competition}
\cvitem[2001]{Instituto Superior Técnico (IST) Academic Excellence Award}

\break
\section{Teaching Experience}

\cvitem[2013--2014]{Software Carpentry}
I am currently an official Software Carpentry instructor and have taught at
several workshops in Europe and abroad.

\cvitem[2012]{Programming for Scientists (short course)}
A two-and-a-half day course introductory programming course for scientists. Due
to overwhelming demand from students, I taught two sessions.

\cvitem[2012]{Practical Tutorial in Next Generation Sequencing in ``Omics for Parasite Biology'' Course}
\cvitem[2011--2013]{Practical Tutorial in Lisbon Machine Learning Summer School}

\cvitem[2009]{Programming for Scientists}
I designed and taught a semester-long course on computer programming for
scientists at Carnegie Mellon University. This course was not based on any
existing course at CMU, but resulted from a need I identified in biomedical or
physical scientists who, as part of their job, have a need to program.

\cvitem[2008]{Teaching Assistant for Laboratory Methods for Computational Biologists (CMU)}
\cvitem[2005]{Introduction to Computers Course in Cacém}
I co-developed and co-taught an introductory course in computer usage in
Cacém, an underprivileged neighborhood near Lisbon. This was pro-bono work.

\cvitem[2005]{Teaching Assistant for Decision Support Systems (IST)}

\section{Organisational \& Mentoring Experience}
\cvitem[2012--2013]{Member of the Organization of the Lisbon Machine Learning School}

\cvitem[2010--2012]{Beira Project \& Rabbit Bounce}
With Rita Reis, I started the Beira Project. We spent two months as volunteers
in Beira (Mozambique) working with organisations that provide public health
information and services, mostly relating to \textsc{hiv}. We had also held
organised successful fund-raising activities for our partner organisations in
Mozambique. In 2012, we started a non-profit association called \textit{Rabbit
Bounce}, which will raise money for education in Mozambique.

\cvitem[2012--2013]{Supervision of Students on Computational Projects}
I am currently advising a Master thesis in Computer Science (Paulo Monteiro)
and co-advising another in Biomedical Engineering (Joana Cruz). During 2012, I
was co-mentor two bioinformaticians in Luisa M.\ Figueiredo's lab as part of a
collaboration between labs.

\cvitem[2008--2011]{Mentoring Junior Members of Murphy Lab}
During my PhD.\ studies, I had the opportunity to directly supervise and mentor
several junior members of the Murphy Lab. This experience includes working with
\textbf{paid undergraduate programmers} (Nathan Herzing and Jephthah
Liddie\ignore{---both currently still students at Carnegie Mellon University}),
one \textbf{MSc.\ student} (Shannon Quinn\ignore{, currently a doctoral student
in the Joint Carnegie Mellon University--University of Pittsburgh PhD.\ Program
in Computational Biology}), \textbf{undergraduate students} performing lab work
for credit (Jimmy Xu), and \textbf{high school students} volunteering over the
summer (Peter Webb and Robert Webb).

\cvitem[2010]{Local Committee for Portuguese-American Postgraduate Society National Forum}
I headed the local organising committee for the 2010 edition of this annual
event. It took place in Pittsburgh and included, as speakers, cabinet-level
Portuguese government officials, renowned researchers, artists, as well as
participants from all around the US.

\cvitem[2002--2004]{Producer for IST Theatre Group}
I served as the producer for the IST Theatre Group, which is one of the top
university theatre groups in Portugal. We participated in several festivals,
including international festivals. As producer, my activities included
fund-raising and management.

\section{Other Experience}

\cvitem{Open Source Software}
I have released several important open source packages related to my research
work, namely in the areas of \textbf{machine learning}, which is addressed by
the packages \textit{milk} and \textit{elgreco}; and \textbf{computer vision},
which is addressed by the package \textit{mahotas}. Previously, I was a
\textsc{kde} developer. The \textsc{kde} project is one of the largest
open-source projects in the world with several hundred developers and over one
million lines of code.

\cvitem{Peer Review}
I have reviewed papers for BMC~Bioinformatics, PLoS Computational Biology,
Database, IEEE~Transactions in Computational Biology and~Bionformatics, and
IEEE~Transactions on Medical Imaging. I have also reviewed grant proposals for
the Czech Science Foundation.

%\section{Skills}
%
%\cvitem{Computer Programming}
%I am an experienced programmer in several languages including \textbf{C},
%\textbf{C\raisebox{.2em}{\tiny \bf ++}}, \textbf{Python}, \textbf{Matlab}, and
%\textbf{Haskell}.
%
%\cvitem{Cell Culture \& Fluorescent Imaging}
%As part of my doctoral research, I was responsible for \textbf{fluorescent
%imaging} of \textbf{cultured mammalian cells}. During my postdoctoral research,
%I extended this expertise to culturing blood-stage Plasmodium falciparum.
%
\cvitem{Language Skills}
I am bilingual in \textbf{English} and \textbf{Portuguese}. I speak and write
fluent \textbf{German} (I attended a German high-school, obtaining an Abitur;
and later spent a year as an exchange student at Technical University of
Vienna) and speak fluent \textbf{French}. I know basic \textbf{Luxembourgish}.

\pagebreak
\section{Full List of Publications}

\subsection{Peer-Reviewed Research Papers}
\begin{enumerate}

\item \textbf{Luis Pedro Coelho}, Catarina Pato, Ana Friães, Ariane Neumann,
Maren von Köckritz-Blickwede, Mário Ramirez, João André Carriço,
\emph{Automatic determination of NET (neutrophil extracellular traps) coverage
in fluorescent microscopy images} in Bioinformatics (2015, accepted)
\showdoi{10.1093/bioinformatics/btv156}

\item Ana C. Pena, Mafalda R. Pimentel, Helena Manso, Rita Vaz-Drago, Daniel
Pinto-Neves, Francisco Aresta-Branco, Filipa Rijo-Ferreira, Fabien Guegan,
\textbf{Luis Pedro Coelho}, Maria Carmo-Fonseca, Nuno L. Barbosa-Morais, Luisa
M. Figueiredo, \emph{Trypanosoma brucei histone H1 inhibits RNA polymerase I
transcription and is important for parasite fitness in vivo} in Molecular
Microbiology (2014) \showdoi{10.1111/mmi.12677}

\item Kristoffer Forslund, Shinichi Sunagawa, \textbf{Luis P. Coelho}, Peer
Bork, \emph{Metagenomic insights into the human gut resistome and the forces that
shape it} in Bioessays (2014) \showdoi{10.1002/bies.201300143}

\item  Peter Liehl,  Vanessa Zuzarte-Luís,  Jennie Chan,  Thomas Zillinger,
Fernanda Baptista,  Daniel Carapau,  Madlen Konert, Kirsten K Hanson, Céline
Carret,  Caroline Lassnig,  Mathias Müller,  Ulrich Kalinke, Mohsan Saeed,
Angelo Ferreira Chora,  Douglas T Golenbock,  Birgit Strobl, Miguel Prudêncio,
\textbf{Luis P Coelho},  Stefan H Kappe,  Giulio Superti-Furga, Andreas
Pichlmair,  Ana M Vigário,  Charles M Rice, Katherine A Fitzgerald, Winfried
Barchet, and Maria M Mota, \emph{Host-cell sensors for Plasmodium activate
innate immunity against liver-stage infection} in Nature Medicine 20, 47-53
(2014) \showdoi{10.1038/nm.3424}

\item Shinichi Sunagawa,  Daniel R Mende,  Georg Zeller,  Fernando
Izquierdo-Carrasco,  Simon A Berger,  Jens Roat Kultima,  \textbf{Luis Pedro
Coelho}, Manimozhiyan Arumugam,  Julien Tap, Henrik Bjørn Nielsen,  Simon
Rasmussen, Søren Brunak,  Oluf Pedersen,  Francisco Guarner, Willem M de Vos,
Jun Wang,  Junhua Li,  Joël Doré,  S Dusko Ehrlich,  Alexandros Stamatakis and
Peer Bork, \emph{Metagenomic species profiling using universal phylogenetic
marker genes} in Nature Methods, 2013 \showdoi{10.1038/nmeth.2693}

\item \textbf{Luis Pedro Coelho}, Joshua D.  Kangas, Armaghan Naik, Elvira
Osuna-Highley, Estelle Glory-Afshar, Margaret Fuhrman, Ramanuja Simha, Peter B.
Berget, Jonathan W.  Jarvik, and Robert F. Murphy, \emph{Determining the
subcellular location of new proteins from microscope images using local
features} in Bioinformatics, 2013 \showdoi{10.1093/bioinformatics/btt392}

\item \textbf{Luis Pedro Coelho} Mahotas: Open source software for scriptable
computer vision, Journal of Open Research Software, vol.\ 1, 2013
\showdoi{10.5334/jors.ac}

\item \textbf{Luis Pedro Coelho}, Tao Peng, and Robert F. Murphy,
\emph{Quantifying the distribution of probes between subcellular locations
using unsupervised pattern unmixing} in Bioinformatics, vol.\ 26 (12), pp.\
i7--i12, 2010 \showdoi{10.1093/bioinformatics/btq220}

\item \textbf{Luis Pedro Coelho}, Amr Ahmed, Andrew Arnold, Joshua Kangas,
Abdul-Saboor Sheikh, Eric P. Xing, William W. Cohen, and Robert F. Murphy,
\emph{Structured Literature Image Finder: Extracting Information from Text and
Images in Biomedical Literature} in Lecture Notes in Bioinformatics, vol.\
6004, pp.\ 23--32, 2010 \showdoi{10.1007/978-3-642-13131-8_4}

\item Amr Ahmed, Andrew Arnold, \textbf{Luis Pedro Coelho}, Joshua Kangas,
Abdul-Saboor Sheikk, Eric P. Xing, William W. Cohen, \emph{Structured
Literature Image Finder: Parsing Text and Figures in Biomedical Literature} in
Web Semantics: Science, Services and Agents on the World Wide Web, vol.\ 8,
pp.\ 151--154, 2010 \showdoi{10.1016/j.websem.2010.04.002}

\item Amr Ahmed, Andrew Arnold, \textbf{Luis Pedro Coelho}, Joshua Kangas,
Abdul-Saboor Sheikk, Eric P. Xing, William W. Cohen, and Robert F. Murphy;
\emph{Structured Literature Image Finder}, Proceedings of BioLINK (ISMB Special
Interest Group), 2009

\item \textbf{Luis Pedro Coelho}, Aabid Shariff, and Robert F. Murphy;
\emph{Nuclear segmentation in microscope cell images: A hand-segmented dataset
and comparison of algorithms} in Proceedings of IEEE International Symposium in
Biomedical Imaging, 2009 \showdoi{10.1109/ISBI.2009.5193098}

\item Taraz Buck, Arvind Rao, \textbf{Luis Pedro Coelho}, Margaret Fuhrman,
Jonathan W. Jarvik, Peter B. Berget, and Robert F. Murphy; \emph{Cell Cycle
Dependence of Protein Subcellular Location Inferred from Static, Asynchronous
Images} in Conference Proceedings of the IEEE Engineering in Medical and
Biology Society, pp. 1016--1019, 2009 \showdoi{10.1109/IEMBS.2009.5332888}

\item \textbf{Luis Pedro Coelho} and Robert Murphy; \emph{Identifying
Subcellular Locations from Images of Unknown Resolution} in Bioinformatics
Research and Development Communications in Computer and Information Science,
vol.\ 13, pp.\ 235--242, 2008 \showdoi{10.1007/978-3-540-70600-7_18}

\item Amina Chebira, \textbf{Luis Pedro Coelho}, Aliaksei Sandryhaila, Stephen
Lin, William G. Jenkinson, Jeremiah MacSleyne, Christopher Hoffman, Philipp
Cuadra, Charles Jackson, Markus Püschel, Jelena Kovacevick; \emph{An Adaptive
Multiresolution Approach to Fingerprint Recognition} in Proceedings of IEEE
International Conference on Image Processing, 2007
\showdoi{10.1109/ICIP.2007.4378990}

\item \textbf{Luis Pedro Coelho} and Arlindo Oliveira; \emph{Dotted Suffix
Trees: A Structure for Approximate Text Indexing} in String Processing and
Information Retrieval Lecture Notes in Computer Science, vol.\ 4209, pp.\
329--336, 2006 \showdoi{10.1007/11880561_27}
\end{enumerate}

\subsection{Review Articles}
\begin{enumerate}
\item \textbf{Luis Pedro Coelho}, Estelle Glory-Afshar, Joshua Kangas, Shannon
Quinn, Aabid Shariff, and Robert F. Murphy; \emph{Principles of Bioimage
Informatics: Focus on machine learning of cell patterns} in Linking Literature,
Information, and Knowledge for Biology Lecture Notes in Computer Science, vol.
6004, pp. 8--18, 2010 \showdoi{10.1007/978-3-642-13131-8_2}

\item Aabid Shariff, Joshua Kangas, \textbf{Luis Pedro Coelho}, Shannon Quinn,
and Robert F. Murphy; \emph{Automated Image Analysis for High Content Screening
and Analysis} in Journal Biomolecular Screening, August 2010, pp.\ 726--734
\showdoi{10.1177/1087057110370894}
\end{enumerate}
\subsection{Books}
\begin{enumerate}
\item Willi Richert, \textbf{Luis Pedro Coelho}; \emph{Building Machine
Learning Systems with Python}, Packt Publishing, 2013
\end{enumerate}

\subsection{Invited Talks}
\begin{enumerate}
\item \emph{Python for Computer Vision in Biology and Beyond}, Python San
    Sebastian, September 2014
\item \emph{Large Scale Analysis of Bioimages Using Python}, International
    Workshop on Technical Computing for Machine Learning and Mathematical
    Engineering, Leuven (Belgium), September 2014
\item \emph{Organizing the Proteome with Location and Function Topics},
Freiburg Institute for Advanced Studies, April 2013
\item \emph{Modeling Subcellular Location from Images and Other Sources of
Information}, Luxembourg Center for Systems Biology, July 2012
\item \emph{Modeling Subcellular Location from Images and Other Sources of
Information}, EAO Seminar (Instituto Gulbenkian da Ciência), Oeiras, March 2012
\item \emph{Learning Subcellular Location from Images and Other Sources of
Information}, KDBIO (Knowledge Discovery and Bioinformatics) seminar, Lisbon,
February 2012
\item \emph{Bioimage Informatics: Computer Vision for Biology}, EMBO Practical
Course on Microscopy: from single molecules to animals, Pretoria, November 2011
\item \emph{Studying the subcellular location space with bioimages and other
data modalities}, University of Delaware, Computer and Information Sciences
Department, September 2011
\item \emph{Unsupervised Mixture Pattern Unmixing}, University of Bielefeld
International Graduate School of Bioinformatics and Genome Research, July 2008
\item \emph{Proteome-scale analysis and modeling of subcellular location}, 4th
CeBiTec Symposium BioImaging, Bielefeld (Germany), 25--27 August 2009
\end{enumerate}

\subsection{Other Talks}

\begin{enumerate}
\item Rita Reis and \textbf{Luis Pedro Coelho}; \emph{Using Theatre to Fight
HIV/AIDS in Mozambique}, National Conference of the Association for Theatre in
Higher Education, Chicago 2011
\item \textbf{Luis Pedro Coelho} and Robert F. Murphy; \emph{Determining
Resolvable Subcellular Location Categories as a Function of Image Resolution},
24th ISAC Congress, Budapest, May 2008
\end{enumerate}

\end{document}
