\documentclass{article}
\usepackage[utf8]{inputenc}
\usepackage{textcomp}
\usepackage{graphicx}
\usepackage{xcolor}

\definecolor{lightg}{HTML}{999999}
\definecolor{medg}{HTML}{666666}
\definecolor{darkg}{HTML}{333333}

\usepackage{wrapfig}
\usepackage[T1]{fontenc}
\usepackage{cmbright}
%FONTS
\usepackage{fontspec}
\font\head="Qlassik Medium:letterspace=4" at 34pt % http://www.dafont.com/qlassik.font
\font\subhead="Qlassik Medium" at 15pt
\font\subsubhead="Qlassik Medium" at 15pt
\font\Quote="Linux Libertine O Bold Italic:mapping=tex-text" at 16pt
\font\Text="Linux Libertine O" at 11pt
\font\Textit="Linux Libertine O/I:+onum" at 11pt 	% /I select Italic
\font\Textblack="Linux Libertine O/BI" at 9pt 	% /BI select Bold Italic 
\font\Textsc="Linux Libertine O:+smcp" at 11pt 	% +smcp selects Small Caps
\font\wdate="Linux Libertine O:+smcp, letterspace=4,+onum" at 9pt % +onum old style numerals

\renewcommand\section[1]{%
    \par\vspace{2em}%
    {\subhead #1}%
    \par\vspace{1em}%
}
\renewcommand\subsection[1]{%
    \par\vspace{.1em}%
    {\hspace{1em}\subsubhead #1}%
    \par\vspace{.2em}%
}
\newcommand\cvitem[2][\relax]{%
    \par\vspace{.8em}
    \if\relax#1\else{\Textsc #1}\hspace{1em}\fi%
    {\Textblack #2}%
    \par\vspace{.4em}
}

\usepackage[margin=1in]{geometry}
\usepackage[pdftex,bookmarks,colorlinks,breaklinks]{hyperref}  % PDF 

%Styling Itemizations
\usepackage{enumitem}
\renewcommand{\labelitemi}{\textcolor{lightg}{\symbol{"F6B9}}}
\begin{document}

\parindent=0cm

\begin{minipage}[t]{0.60\linewidth}%
\head \textcolor{darkg}{LUIS PEDRO COELHO}

\begin{minipage}[t]{0.70\linewidth}%
\raggedright\Quote\color{darkg} A computational biologist interested in the cell biology the malaria Plasmodium
\end{minipage}
\end{minipage}
\hfill
\begin{minipage}[t]{0.26\textwidth}%
\vspace{-2.2em}

\Text \textcolor{medg}{
Carnegie Mellon University\\
5000 Forbes Ave, GHC 7405\\
Pittsburgh, PA (USA)}\\
\textcolor{darkg}{phone} : (+1)-412-330-8306\\
\textcolor{darkg}{email} : lpc@cmu.edu
\end{minipage}

\vspace{2.3em}

\section{Research Highlights}

\cvitem[2010]{Unsupervised Subcellular Pattern Unmixing}
Many proteins are located in more than one organelle simultaneously, a
phenomenon known as a mixed pattern (the pattern corresponding to a single
organelle being a pure pattern). Recovering the organelle structure from image
data where mixed patterns are present is known as the unsupervised unmixing
problem in subcellular location. I developed new models for
solving this problem. My solution was chosen as one of the highlights of 2010
in computational biology by the journal \emph{Nature}.

\cvitem[2008-2010]{Structure Literature Image Finder (SLIF)}
SLIF is a search engine that indexes biomedical papers with their images. In
this project, I was responsible for the computer vision aspects of the
processing pipeline. I also handled much of the integration effort of the
several components and coordinated the preparation of the multi-author
publications that resulted from the effort. This project was a finalist in the
Elsevier Grand Challenge (4 teams out of 70 were chosen for the final). I
represented our team in both the final and the semi-final of this competition.

\section{Education}

\cvitem[Summer 2011]{PhD in computational biology, Carnegie Mellon University}
Advised by Prof.\ Robert F. Murphy, I presented new automated techniques for
the proteome-wide subcellular localisation of proteins in mouse cells based on
fluorescent images and other data

\cvitem[2006]{MS in computer science, Instituto Superior T\'ecnico (Technical University Lisbon)}
Advised by Prof.\ Arlindo Oliveira, I analysed the theoretical limitations of
extracting information from noisy data.

\cvitem[2004]{BS in computer science, Instituto Superior T\'ecnico (Technical University Lisbon)}
One year spent in the Technical University of Vienna as an exchange student.
Finished top my class.

\section{Teaching Experience}

\cvitem[2011]{Practical Tutorial in Lisbon Machine Learning Summer School}

\cvitem[2009]{Programming for Scientists}
I designed and taught a semester-long course on computer programming for
scientists at Carnegie Mellon University. This course was not based on any
existing course at CMU, but resulted from a need I identified in biomedical or
physical scientists who, as part of their job, have a need to program.

\cvitem[2008]{Teaching Assistant for Laboratory Methods for Computational Biologists (CMU)}
\cvitem[2005]{Introduction to Computers Course in Cac\'em}
I co-developed and co-taught an introductory course in computer usage in
Cac\'em, an underprivileged neighbourhood near Lisbon. This was pro-bono work.

\cvitem[2005]{Teaching Assistant for Decision Support Systems (IST)}

\section{Other Research Projects}

\cvitem[2005-2006]{Approximate String Indexing}
An approximate string index is a data structure that allows for fast string
queries where the matches may be approximate. I developed one such index, the
\emph{dotted suffix tree}.

\section{Honours \&\ Awards}

\cvitem[2007--2011]{PhD. Scholarship from Portuguese Science Foundation}
Grant for PhD.\ studies in the United States.

\cvitem[2009]{Research Excellence Award}
From the Joint Carnegie Mellon--University of Pittsburgh Ph.D. Program in Computational Biology.

\cvitem[2008]{Academic Excellence Award}
From the Joint Carnegie Mellon--University of Pittsburgh Ph.D. Program in Computational Biology.

\cvitem[2006]{Fulbright Fellow}
Grant for PhD.\ studies in the United States.

\cvitem[2005--2006]{Scholarship from Portuguese Science Foundation}
Grant for introduction to research.

\cvitem[2001]{Academic Excellence Award}
From Instituto Superior T\'ecnico.

\section{Organisational Experience}

\cvitem[2010]{Local Committee for Portuguese-American Postgraduate Society National Forum}
I headed the local organising committee for this the 2010 edition of this
annual event. It took place in Pittsburgh and included, as speakers,
cabinet-level Portuguese government officials, renowned researchers, artists,
as well as participants from all around the US.

\cvitem[2010]{Beira Project}
With Rita Reis, I started the Beira Project. We spent two months as volunteers
in Beira, Mozambique. We had previously held several successful fund-raising
activies for organisations in Mozambique, raising almost 4000 USD.

\cvitem[2002--2004]{Producer for IST Theatre Group}
I acted as the producer for the IST Theatre Group, which is one of the top
university theatre groups in Portugal. We participated in several festival,
including international festivals. As producer, my activities included
fund-raising and management.

\section{Other Experience}

\cvitem{Open Source Software}
I have released several important open source packages related to my research
work, namely in the areas of machine learning and computer vision. Previously,
I was a \textsc{kde} developer. The \textsc{kde} project is one of the largest
open-source projects in the world with several hundred developers and over one
million lines of code.

\section{Skills}

\cvitem{Computer Programming}
I am an experienced programmer in several languages including \textbf{C},
\textbf{C\raisebox{.1em}{\tiny \bf ++}}, \textbf{Python}, \textbf{Matlab}, and
\textbf{Haskell}.

\cvitem{Cell Culture \& Imaging}
As part of my doctoral research, I was responsible for imaging cells and basic
cell culture.

\cvitem{Language Skills}
I am bilingual in English and Portuguese. I speak and write fluent German (I
attended a German high-school and the Technical University of Vienna) and speak
fluent French (I have lived for a in France).

\vspace{4.1em}
\section{Full List of Publications}
\vspace{2em}

\subsection{Peer-Reviewed Journal Articles}
\begin{enumerate}
\item \textbf{Luis Pedro Coelho}, Tao Peng, and Robert F. Murphy,
\emph{Quantifying the distribution of probes between subcellular locations
using unsupervised pattern unmixing} in Bioinformatics (2010) 26 (12): i7-i12.
\item \textbf{Luis Pedro Coelho}, Amr Ahmed, Andrew Arnold, Joshua Kangas,
Abdul-Saboor Sheikh, Eric P. Xing, William W. Cohen, and Robert F. Murphy,
\emph{Structured Literature Image Finder: Extracting Information from Text and
Images in Biomedical Literature} in Lecture Notes in Bioinformatics vol. 6004
\item \textbf{Luis Pedro Coelho}, Estelle Glory-Afshar, Joshua Kangas, Shannon
Quinn, Aabid Shariff, and Robert F. Murphy; \emph{Principles of Bioimage
Informatics: Focus on machine learning of cell patterns} in Lecture Notes in
Computer Science
\item Amr Ahmed, Andrew Arnold, \textbf{Luis Pedro Coelho}, Joshua Kangas,
Abdul-Saboor Sheikk, Eric P. Xing, William W. Cohen, \emph{Structured
Literature Image Finder: Parsing Text and Figures in Biomedical Literature} in
Web Semantics: Science, Services and Agents on the World Wide Web, 2010
\item Aabid Shariff, Joshua Kangas, \textbf{Luis Pedro Coelho}, Shannon Quinn,
and Robert F. Murphy; \emph{Automated Image Analysis for High Content Screening
and Analysis}, Journal Biomolecular Screening (2010)
\end{enumerate}

\subsection{Peer-Reviewed Conference Papers}
\begin{enumerate}
\item \textbf{Luis Pedro Coelho} and Arlindo Oliveira; \emph{Dotted Suffix
Trees: A Structure for Approximate Text Indexing} SPIRE (2006)
\item Amina Chebira, \textbf{Luis Pedro Coelho}, Aliaksei Sandryhaila, Stephen
Lin, William G. Jenkinson, Jeremiah MacSleyne, Christopher Hoffman, Philipp
Cuadra, Charles Jackson, Markus P\"uschel, Jelena Kovacevick; \emph{An Adaptive
Multiresolution Approach to Fingerprint Recognition}, International Conference
on Image Processing, (2007)
\item \textbf{Luis Pedro Coelho} and Robert Murphy; \emph{Identifying
Subcellular Locations from Images of Unknown Resolution} Bioinformatics
Research and Development, LNCS, Springer, Volume 13, Vienna, Austria (2008)
\item \textbf{Luis Pedro Coelho} and Robert F. Murphy; \emph{Unsupervised
Unmixing of Subcellular Location Patterns}, Proceedings of ICML-UAI-COLT 2009
Workshop on Automated Interpretation and Modeling of Cell Images (2009).
\item Amr Ahmed, Andrew Arnold, \textbf{Luis Pedro Coelho}, Joshua Kangas,
Abdul-Saboor Sheikk, Eric P. Xing, William W. Cohen, and Robert F. Murphy;
\emph{Structured Literature Image Finder}, Proceedings of BioLINK (ISMB Special
Interest Group 2009).
\item \textbf{Luis Pedro Coelho}, Aabid Shariff, and Robert F. Murphy;
\emph{Nuclear segmentation in microscope cell images: A hand-segmented dataset
and comparison of algorithms} ISBI 2009.
\item Taraz Buck, Arvind Rao, \textbf{Luis Pedro Coelho}, Margaret Fuhrman,
Jonathan W. Jarvik, Peter B. Berget, and Robert F. Murphy; \emph{Cell Cycle
Dependence of Protein Subcellular Location Inferred from Static, Asynchronous
Images} EMBC 2009.
\item Rita Reis and \textbf{Luis Pedro Coelho}, \emph{Using Theatre to Fight
HIV/AIDS in Mozambique}, National Conference of the Association for Theatre in
Higher Education, Chicago 2011
\end{enumerate}


\subsection{Invited Talks}
\begin{enumerate}
\item \emph{Proteome-scale analysis and modeling of subcellular location},4th
CeBiTec Symposium BioImaging, Bielefeld (Germany) 25--27 August 2009
\item \emph{Unsupervised Mixture Pattern Unmixing}, University of Bielefeld
International Graduate School of Bioinformatics and Genome Research, July 2008.
\end{enumerate}


\subsection{Other Talks}

\begin{enumerate}
\item \emph{Determining Resolvable Subcellular Location Categories as a
Function of Image Resolution}, \textbf{Luis Pedro Coelho} and Robert F. Murphy,
24th ISAC Congress, Budapest, May 2008.
\end{enumerate}

\end{document}
