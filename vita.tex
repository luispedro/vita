\documentclass{article}
\usepackage{textcomp}
\usepackage{graphicx}
\usepackage{xcolor}

\definecolor{lightg}{HTML}{999999}
\definecolor{medg}{HTML}{666666}
\definecolor{darkg}{HTML}{333333}

\usepackage{wrapfig}
\usepackage{cmbright}
%FONTS
\usepackage{fontspec}
\font\head="Qlassik Medium:letterspace=4" at 34pt % http://www.dafont.com/qlassik.font
\font\subhead="Avdira Italic" at 13pt
\font\subsubhead="Avdira Italic" at 11pt
\font\Quote="Linux Libertine O Bold Italic:mapping=tex-text" at 16pt
\font\Contact="Linux Libertine O" at 11pt
\font\Text="Linux Libertine O:mapping=tex-text" at 11pt
\font\Textit="Linux Libertine O/I:+onum" at 11pt
\font\CvItem="Lucida Sans Regular:mapping=tex-text" at 10pt
\font\Date="Lucida Sans Demibold:mapping=tex-text" at 9pt

\renewcommand\section[1]{%
    \par\vspace{2em}\penalty-100%
    {\subhead #1}%
    \par\penalty100\vspace{1em}\penalty100%
}
\renewcommand\subsection[1]{%
    \par\vspace{.1em}%
    {\hspace{1em}\subsubhead #1}%
    \par\vspace{.2em}%
}
\newcommand\cvitem[2][\relax]{%
    \par\vspace{.8em}
    \if\relax#1\else{\Date \textcolor{medg}{#1}}\hspace{1em}\fi%
    {\CvItem #2}%
    \par\vspace{.4em}
}
\newcommand\showdoi[1]{%
    \href{http://dx.doi.org/#1}{[DOI]}%
}
\newcommand\ignore[1]{\relax}

\usepackage[margin=1in]{geometry}
\usepackage[pdftex,bookmarks,colorlinks,breaklinks]{hyperref}

\usepackage{enumitem}
\renewcommand{\labelitemi}{\textcolor{lightg}{\symbol{"F6B9}}}
\pagestyle{empty}
\begin{document}

\parindent=0cm

\begin{minipage}[t]{0.60\linewidth}%
\head \textcolor{darkg}{LUIS PEDRO COELHO}

\end{minipage}
\hfill
\begin{minipage}[t]{0.30\textwidth}%
\vspace{-2.2em}
{\Contact%
\textcolor{medg}{
Carnegie Mellon University\\
5000 Forbes Ave, GHC 7405\\
Pittsburgh, PA (USA)}\\
\textcolor{darkg}{phone} : (+1)-412-330-8306\\
\textcolor{darkg}{email} : lpc@cmu.edu

\textcolor{darkg}{Citizenship} : U.K.\ and Portugal
}
\end{minipage}

\vspace{2.3em}

\section{Research Highlights}
\Text

\cvitem[2011]{Proteome-wide Study of Subcellular Location in Mouse Fibroblasts}
I am developing new methods for learning from both a very large collection of
images of fluorescently tagged proteins and other forms of data (sequence
information, database information). This work builds on my past work and is, at
the moment, still unpublished.

\cvitem[2010]{Unsupervised Subcellular Pattern Unmixing}
Many proteins are located in more than one organelle simultaneously, a
phenomenon known as a mixed pattern (the pattern corresponding to a single
organelle being a pure pattern). Recovering the organelle structure from image
data where mixed patterns are present is known as the unsupervised unmixing
problem in subcellular location. I developed new models for
solving this problem. My solution was chosen as one of the highlights of 2010
in computational biology by the journal \emph{Nature Biotechnology}.\footnote{
\emph{Trends in computational biology—2010} by H. Craig Mak, Nature
Biotechnology, vol.\ 29(1), Jan.\ 2011, pp.\ 45--45 \showdoi{10.1038/nbt.1747}}

\cvitem[2008--2009]{Structure Literature Image Finder (SLIF)}
SLIF is a search engine that indexes biomedical papers with their images. In
this project, I was responsible for the computer vision aspects of the
processing pipeline. I also handled much of the integration effort of the
several components and coordinated the preparation of the multi-author
publications that resulted from the effort. This project was a finalist in the
Elsevier Grand Challenge (4 teams out of 70 were chosen for the final). I
represented our team in both the final and the semi-final of this competition.

\section{Education}

\cvitem[Summer 2011]{PhD in computational biology, Carnegie Mellon University}
Dissertation topic: \emph{Modeling the Space of Subcellular Location Patterns
Using Images and Other Sources of Information}, advised by Prof.\ Robert F.
Murphy.

\cvitem[2006]{MS in computer science, Instituto Superior Técnico (Technical University Lisbon)}
Dissertation topic: \emph{Bayesian Network Parameter Estimation Using Noisy
Observations or Soft Evidence}, advised by Prof.\ Arlindo Oliveira.

\cvitem[2004]{BS in computer science, Instituto Superior Técnico (Technical University Lisbon)}
Finished top of my class.

\section{Teaching Experience}

\cvitem[2011]{Practical Tutorial in Lisbon Machine Learning Summer School}

\cvitem[2009]{Programming for Scientists}
I designed and taught a semester-long course on computer programming for
scientists at Carnegie Mellon University. This course was not based on any
existing course at CMU, but resulted from a need I identified in biomedical or
physical scientists who, as part of their job, have a need to program.

\cvitem[2008]{Teaching Assistant for Laboratory Methods for Computational Biologists (CMU)}
\cvitem[2005]{Introduction to Computers Course in Cacém}
I co-developed and co-taught an introductory course in computer usage in
Cacém, an underprivileged neighbourhood near Lisbon. This was pro-bono work.

\cvitem[2005]{Teaching Assistant for Decision Support Systems (IST)}

\section{Other Research Projects}

\cvitem[2006]{Parameter Estimation in Bayesian Networks from Noisy Data}
A Bayesian network is a fundamental class of graphical probabilistic models and
learning parameters from data is a fundamental problem. In this work, I
investigated the theoretical limits of learning from noisy data.

\cvitem[2005]{Approximate String Indexing}
An approximate string index is a data structure that allows for fast string
queries where the matches may be approximate. I developed one such index, the
\emph{dotted suffix tree}.

\section{Scholarships \&\ Awards}

\cvitem[2007--2011]{PhD. Scholarship from Portuguese Science Foundation}
\cvitem[2009]{Joint CMU-U. of Pittsburgh PhD.\ in Computational Biology Research Excellence Award}
\cvitem[2008]{Joint CMU-U. of Pittsburgh PhD.\ in Computational Biology Academic Excellence Award}
\cvitem[2006]{Fulbright Fellow}
\cvitem[2005]{Scholarship from Portuguese Science Foundation}
\cvitem[2004]{Second Prize in \emph{Lisboa à Letra} short story competition}
\cvitem[2001]{Instituto Superior Técnico (IST) Academic Excellence Award}

\section{Mentoring Experience}
\cvitem[2008--2011]{Mentoring Junior Members of Murphy Lab}
During my PhD.\ studies, I had the opportunity to directly supervise and mentor
several junior members of the Murphy Lab. This experience includes working with
\textbf{paid undergraduate programmers} (Nathan Herzing and Jephthah
Liddie\ignore{---both currently still students at Carnegie Mellon University}),
one \textbf{MSc.\ student} (Shannon Quinn\ignore{, currently a doctoral student
in the Joint Carnegie Mellon University--University of Pittsburgh PhD.\ Program
in Computational Biology}), \textbf{undergraduate students} performing lab work
for credit (Jimmy Xu), and \textbf{high school students} volunteering over the
summer (Peter Webb and Robert Webb).

\section{Organisational Experience}

\cvitem[2010]{Local Committee for Portuguese-American Postgraduate Society National Forum}
I headed the local organising committee for this the 2010 edition of this
annual event. It took place in Pittsburgh and included, as speakers,
cabinet-level Portuguese government officials, renowned researchers, artists,
as well as participants from all around the US.

\cvitem[2010]{Beira Project}
With Rita Reis, I started the Beira Project. We spent two months as volunteers
in Beira (Mozambique) working with organisations that provide public health
information and services, mostly relating to \textsc{hiv}. We had also held
organised successful fund-raising activities for our partner organisations in
Mozambique.

\cvitem[2002--2004]{Producer for IST Theatre Group}
I served as the producer for the IST Theatre Group, which is one of the top
university theatre groups in Portugal. We participated in several festivals,
including international festivals. As producer, my activities included
fund-raising and management.

\section{Other Experience}

\cvitem{Open Source Software}
I have released several important open source packages related to my research
work, namely in the areas of \textbf{machine learning}, which is addressed by
the packages \textit{milk} and \textit{elgreco}; and \textbf{computer vision},
which is addressed by the package \textit{mahotas}. Previously, I was a
\textsc{kde} developer. The \textsc{kde} project is one of the largest
open-source projects in the world with several hundred developers and over one
million lines of code.

\section{Skills}

\cvitem{Computer Programming}
I am an experienced programmer in several languages including \textbf{C},
\textbf{C\raisebox{.1em}{\tiny \bf ++}}, \textbf{Python}, \textbf{Matlab}, and
\textbf{Haskell}.

\cvitem{Cell Culture \& Fluorescent Imaging}
As part of my doctoral research, I was responsible for \textbf{fluorescent
imaging} of \textbf{cultured cells}.

\cvitem{Language Skills}
I am bilingual in \textbf{English} and \textbf{Portuguese}. I speak and write
fluent \textbf{German} (I attended a German high-school and the Technical
University of Vienna) and speak fluent \textbf{French} (I have lived in
France).

\bigskip
\pagebreak
\section{Full List of Publications}

\subsection{Peer-Reviewed Journal Articles}
\begin{enumerate}
\item \textbf{Luis Pedro Coelho}, Tao Peng, and Robert F. Murphy,
\emph{Quantifying the distribution of probes between subcellular locations
using unsupervised pattern unmixing} in Bioinformatics, vol.\ 26 (12), pp.\
i7--i12, 2010 \showdoi{10.1093/bioinformatics/btq220}

\item \textbf{Luis Pedro Coelho}, Amr Ahmed, Andrew Arnold, Joshua Kangas,
Abdul-Saboor Sheikh, Eric P. Xing, William W. Cohen, and Robert F. Murphy,
\emph{Structured Literature Image Finder: Extracting Information from Text and
Images in Biomedical Literature} in Lecture Notes in Bioinformatics, vol.\
6004, pp.\ 23--32, 2010 \showdoi{10.1007/978-3-642-13131-8_4}

\item Amr Ahmed, Andrew Arnold, \textbf{Luis Pedro Coelho}, Joshua Kangas,
Abdul-Saboor Sheikk, Eric P. Xing, William W. Cohen, \emph{Structured
Literature Image Finder: Parsing Text and Figures in Biomedical Literature} in
Web Semantics: Science, Services and Agents on the World Wide Web, vol.\ 8,
pp.\ 151--154, 2010 \showdoi{10.1016/j.websem.2010.04.002}

\end{enumerate}

\subsection{Review Articles}
\begin{enumerate}
\item \textbf{Luis Pedro Coelho}, Estelle Glory-Afshar, Joshua Kangas, Shannon
Quinn, Aabid Shariff, and Robert F. Murphy; \emph{Principles of Bioimage
Informatics: Focus on machine learning of cell patterns} in Linking Literature,
Information, and Knowledge for Biology Lecture Notes in Computer Science, vol.
6004, pp. 8--18, 2010 \showdoi{10.1007/978-3-642-13131-8_2}

\item Aabid Shariff, Joshua Kangas, \textbf{Luis Pedro Coelho}, Shannon Quinn,
and Robert F. Murphy; \emph{Automated Image Analysis for High Content Screening
and Analysis} in Journal Biomolecular Screening, August 2010, pp.\ 726--734
\showdoi{10.1177/1087057110370894}

\end{enumerate}

\subsection{Peer-Reviewed Conference Papers}
\begin{enumerate}
\item Amina Chebira, \textbf{Luis Pedro Coelho}, Aliaksei Sandryhaila, Stephen
Lin, William G. Jenkinson, Jeremiah MacSleyne, Christopher Hoffman, Philipp
Cuadra, Charles Jackson, Markus Püschel, Jelena Kovacevick; \emph{An Adaptive
Multiresolution Approach to Fingerprint Recognition} in Proceedings of IEEE
International Conference on Image Processing, 2007
\showdoi{10.1109/ICIP.2007.4378990}

\item \textbf{Luis Pedro Coelho} and Robert Murphy; \emph{Identifying
Subcellular Locations from Images of Unknown Resolution} in Bioinformatics
Research and Development Communications in Computer and Information Science,
vol.\ 13, pp.\ 235--242, 2008 \showdoi{10.1007/978-3-540-70600-7_18}

\item \textbf{Luis Pedro Coelho} and Robert F. Murphy; \emph{Unsupervised
Unmixing of Subcellular Location Patterns}, Proceedings of ICML--UAI--COLT 2009
Workshop on Automated Interpretation and Modeling of Cell Images, 2009

\item Amr Ahmed, Andrew Arnold, \textbf{Luis Pedro Coelho}, Joshua Kangas,
Abdul-Saboor Sheikk, Eric P. Xing, William W. Cohen, and Robert F. Murphy;
\emph{Structured Literature Image Finder}, Proceedings of BioLINK (ISMB Special
Interest Group), 2009

\item \textbf{Luis Pedro Coelho}, Aabid Shariff, and Robert F. Murphy;
\emph{Nuclear segmentation in microscope cell images: A hand-segmented dataset
and comparison of algorithms} in Proceedings of IEEE International Symposium in
Biomedical Imaging, 2009 \showdoi{10.1109/ISBI.2009.5193098}

\item Taraz Buck, Arvind Rao, \textbf{Luis Pedro Coelho}, Margaret Fuhrman,
Jonathan W. Jarvik, Peter B. Berget, and Robert F. Murphy; \emph{Cell Cycle
Dependence of Protein Subcellular Location Inferred from Static, Asynchronous
Images} in Conference Proceedings of the IEEE Engineering in Medical and
Biology Society, pp. 1016--1019, 2009 \showdoi{10.1109/IEMBS.2009.5332888}

\item Rita Reis and \textbf{Luis Pedro Coelho}; \emph{Using Theatre to Fight
HIV/AIDS in Mozambique}, National Conference of the Association for Theatre in
Higher Education, Chicago, 2011

\item \textbf{Luis Pedro Coelho} and Arlindo Oliveira; \emph{Dotted Suffix
Trees: A Structure for Approximate Text Indexing} in String Processing and
Information Retrieval Lecture Notes in Computer Science, vol.\ 4209, pp.\
329--336, 2006 \showdoi{10.1007/11880561_27}
\end{enumerate}


\subsection{Invited Talks}
\begin{enumerate}
\item \emph{Proteome-scale analysis and modeling of subcellular location},4th
CeBiTec Symposium BioImaging, Bielefeld (Germany), 25--27 August 2009
\item \emph{Unsupervised Mixture Pattern Unmixing}, University of Bielefeld
International Graduate School of Bioinformatics and Genome Research, July 2008
\end{enumerate}


\subsection{Other Talks}

\begin{enumerate}
\item \textbf{Luis Pedro Coelho} and Robert F. Murphy; \emph{Determining
Resolvable Subcellular Location Categories as a Function of Image Resolution},
\textbf{Luis Pedro Coelho} 24th ISAC Congress, Budapest, May 2008
\end{enumerate}

\end{document}
